%%%%%%%%%%%%   LaTeX Preamble %%%%%%%%%%%%%%

\documentclass{beamer}

% list all packages
\usepackage{amsmath}
\usepackage{hyperref}
\usepackage{latexsym}
\usepackage{color}
\usepackage{mycommands}
\usepackage{listings}
\lstset{language=Fortran,
        keywordstyle=\color{blue}\textbf,
	commentstyle=\color[rgb]{0.133,0.545,0.133},
	stringstyle=\color[rgb]{0.627,0.126,0.941},
	breaklines=true,
	showstringspaces=false,
	frame=trBL
       }

% slide theme
\usetheme{Berlin}
\usecolortheme{mit}

% Set Logo
\pgfdeclareimage[height=0.5cm]{mit-logo}{../../mit-logo.pdf}
\logo{\vspace{-0.25cm}\pgfuseimage{mit-logo}\hspace*{0.025cm}}

% Include Custom environments
% % \setbeamertemplate{blocks}[rounded]
\setbeamertemplate{blocks}[rounded][shadow=true]

% \setbeamertemplate{headline}[default]
\setbeamertemplate{navigation symbols}{}

% \beamerdefaultoverlayspecification{<+->}

% Set color for 'alert' text
\setbeamercolor{alerted text}{fg=blue}

%\setbeamercolor{section in toc}

% Modify some default font sizes
\setbeamerfont{itemize/enumerate body}{size=\normalfont}
\setbeamerfont{itemize/enumerate subbody}{size=\smaller, shape=\upshape}
\setbeamerfont{frametitle}{size=\large, series=\bfseries}


% \setbeamertemplate{bibliography entry title}{}
% \setbeamertemplate{bibliography entry location}{}
% \setbeamertemplate{bibliography entry note}{}
\setbeamertemplate{bibliography item}[text] 

% \setbeamertemplate{items}[ball]
% \setbeamertemplate{itemize subitem}[circle-symbol]
% \setbeamertemplate{background canvas}[vertical shading][bottom=mitgray!25,top=white]

% Use the shrink option to squeeze lots of text on a slide

% \frame[shrink]{

% …

% }

\colorlet{dark green}{green!50!black}

\newcommand{\packin}{\setlength\abovedisplayskip{2pt}\setlength\belowdisplayskip{2pt}}

%\newcommand{\numberInBox}[2][0.9]%
%	{\scalebox{#1}{{\tikz \draw (0,0) node[refbox] {\makebox[\totalheight]{#2}};}}}

%\newcommand{\enumref}[2][0.9] {\numberInBox[#1]{\ref{#2}}}


% Small arrow pointing down and hooking right
\newcommand{\drarrow}{\scalebox{1.5}{\reflectbox{\rotatebox[c]{180}{$\boldsymbol{\smash[b]{\Rsh}}$}}}}

\newenvironment{prettydescript}[1]
	{\begin{list}{}%
		{\renewcommand\makelabel[1]{\itshape\bfseries\color{mitred} ##1:\hfill}%
		\settowidth\labelwidth{\makelabel{#1}}%
		\setlength\leftmargin{\labelwidth}%
		\addtolength\leftmargin{\labelsep}}}%
	{\end{list}}

\newenvironment{customdescript}[1]
	{\begin{list}{}%
		{\renewcommand\makelabel[1]{\bfseries\color{mitred} ##1\hfill}%
		\settowidth\labelwidth{\makelabel{#1}}%
		\setlength\leftmargin{\labelwidth}%
		\addtolength\leftmargin{\labelsep}}}%
	{\end{list}}

\makeatletter

\newenvironment{customlist}[2]{
  \ifnum\@itemdepth >2\relax\@toodeep\else
      \advance\@itemdepth\@ne%
      \beamer@computepref\@itemdepth%
      \usebeamerfont{itemize/enumerate \beameritemnestingprefix body}%
      \usebeamercolor[fg]{itemize/enumerate \beameritemnestingprefix body}%
      \usebeamertemplate{itemize/enumerate \beameritemnestingprefix body begin}%
      \begin{list}
        {
            \usebeamertemplate{itemize \beameritemnestingprefix item}
        }
        { \leftmargin=#1 \itemindent=#2
            \def\makelabel##1{%
              {%  
                  \hss\llap{{%
                    \usebeamerfont*{itemize \beameritemnestingprefix item}%
                        \usebeamercolor[fg]{itemize \beameritemnestingprefix item}##1}}%
              }%  
            }%  
        }
  \fi
}
{
  \end{list}
  \usebeamertemplate{itemize/enumerate \beameritemnestingprefix body end}%
}
\makeatother

\newenvironment<>{varblock}[2][\textwidth]{%
  \setlength{\textwidth}{#1}
  \begin{actionenv}#3%
    \def\insertblocktitle{#2}%
    \par%
    \usebeamertemplate{block begin}}
  {\par%
    \usebeamertemplate{block end}%
  \end{actionenv}}

%% Notational commands:
\newcommand{\params}{\ensuremath{\xi}}
\newcommand{\vparms}{\ensuremath{\gvect{\params}}}
\renewcommand{\thefootnote}{\ensuremath{\fnsymbol{footnote}}}
\setcounter{footnote}{2}
\renewcommand{\thempfootnote}{\ensuremath{\fnsymbol{mpfootnote}}}
\newcommand{\newsubsection}[1]{\subsection{#1}\setcounter{subsection}{0}}


% Title Page
\title[]{Procedures and Conditionals}
\author[]{22.901 Introduction to Computer Programming for Nuclear Engineers}
\institute[\insertpagenumber]{}
\date{January 23, 2012} 

% -----------------------------------------------------------------------------
\begin{document}
% -----------------------------------------------------------------------------

% Inset title page
\frame{\titlepage}

% Outline slide
\begin{frame}{Outline}
  \begin{itemize}
    \item Subroutines
    \vfill\item Functions
    \vfill\item Logical Operators
  \end{itemize}
\end{frame}
%------------------------------------------------------------------------------
\begin{frame}{Sub-Dividing the Problem}

  \begin{itemize}
    \vfill\item Most programs are \alert{thousands} of lines!
    \vfill\item You may notice you have to have similar code in multiple places
    \vfill\item Want to have good organization
  \end{itemize}
  \begin{center}
    \vfill{\huge\color{blue} use procedures}
  \end{center}

\end{frame}
%------------------------------------------------------------------------------
\begin{frame}{So what can we do?}

  \begin{itemize}
    \item There must be a single {\color{blue}{main}} program
    \vfill\item {\color{blue}{Subroutines}} and {\color{blue}{functions}} are called procedures
    \vfill\item For now we will list them in one file
  \end{itemize}
  \vfill
  Subroutines::
  \begin{itemize}
    \item Code that exists somewhere else
    \item You can pretty much do anything
  \end{itemize}
  \vfill
  Functions::
  \begin{itemize}
    \item Purpose is to return a result 
  \end{itemize}
\end{frame}
%------------------------------------------------------------------------------
\begin{frame}{Calling a subroutine}
  \begin{itemize}
    \item use the call statement, \texttt{call <sub name>(args...)}
  \end{itemize}
  \vfill
  The subroutine is defined somewhere else:
  \begin{itemize}
    \item \texttt{subroutine <name>(args)}
    \item \texttt{use ...}
    \item \texttt{implicit none}
    \item variable declarations
    \item executable statements
    \item \texttt{end subroutine <name>}
  \end{itemize}
\end{frame}
%------------------------------------------------------------------------------
\begin{frame}{Dummy Arguments}

  \begin{itemize}
    \item Variable names in a procedure, only exist in that procedure
    \vfill\item Fortran is \textbf{pass-by-reference}
    \vfill\item The memory location are only passed between procedures (copies are not made!)
    \vfill\item Therefore in arguments list only order matters, not the name!
    \vfill\item Each argument must match in type and rank
  \end{itemize}

\end{frame}
%------------------------------------------------------------------------------
\begin{frame}{Functions}

  \begin{itemize}
    \item Use when the result is a single value
    \vfill\item For example to return the maximum value of a vector
    \vfill\item \textbf{Important:} The function name defines the variable in the function
    \vfill\item The function name is returned at the end of a function
  \end{itemize}
  \vfill
  \texttt{a = maxval(vec,n)}
  \vfill
  \texttt{function maxval(v,n)} \\
  \hspace{0.1cm} \texttt{integer :: n ! length of vector} \\
  \hspace{0.1cm} \texttt{real :: maxval ! the maximum value of a vector} \\
  \hspace{0.1cm} \texttt{real :: vec(n) ! the vector} \\
  \hspace{0.1cm} \texttt{...} \\
  \texttt{end function maxval}

\end{frame}
%------------------------------------------------------------------------------
\begin{frame}{The Intent Attribute}

  \begin{itemize}
    \item You can make arguments read-only, write-only or read-write(default)
    \vfill\item This should be used to avoid overwriting a variable
    \vfill\item It is an attribute so it is listed before the ::
    \vfill\item Example::
      \begin{itemize}
	\item \texttt{integer, intent(in) :: avar ! read-only}
	\item \texttt{integer, intent(out) :: avar ! write-only}
	\item \texttt{integer, intent(inout) :: avar ! read-write}
	\item \texttt{integer :: avar ! read-write}
      \end{itemize}
  \end{itemize}

\end{frame}
%------------------------------------------------------------------------------
\begin{frame}{Internal Procedures}
  
  \begin{itemize}
    \item use the \texttt{contains} command
    \vfill\item this will be use a lot more for modules and object-oriented programming
    \vfill\item included procedure are internal and are private to other routines
    \vfill\item they may not contain their own internal subprograms
  \end{itemize}
  \vfill
  \texttt{program test} \\
  \texttt{...} \\
  \texttt{contains} \\
  \texttt{subroutine atest} \\
  \texttt{...} \\
  \texttt{end subroutine atest} \\
  \texttt{end program test}

\end{frame}
%------------------------------------------------------------------------------
\begin{frame}[allowframebreaks]{Calculating zeta}
  \begin{scriptsize}
    \lstinputlisting{../examples/zeta.f90}
  \end{scriptsize}
\end{frame}
%------------------------------------------------------------------------------
\begin{frame}{Decision Making - Conditionals}

  \begin{itemize}
    \item Almost every time we code, decisions need to be made from with in the code
    \vfill\item Conditionals allow us to execute certain parts of code
    \vfill\item We will go over the following constructs:
    \begin{itemize}
      \item \texttt{if} statement
      \item \texttt{if-then} statement
      \item \texttt{if-else} statement
      \item \texttt{if-elseif-else} statement
    \end{itemize}
  \end{itemize}

\end{frame}
%------------------------------------------------------------------------------
\begin{frame}{The 'if' statement}
  
  \begin{itemize}
    \item Oldest and simplest control statement
    \vfill\item \texttt{if (logical expression) <execute statement>}
    \vfill\item If the LHS is \texttt{.true.}, the RHS is executed
    \vfill\item This is only useful if there is a simple decision and one execute statment
    \vfill\item For anything more complicated, we need to go to block statements
  \end{itemize}
  \vfill
  \begin{center}
    \texttt{if (x < a) x = a}
  \end{center}
\end{frame}
%------------------------------------------------------------------------------
\begin{frame}{The 'if-then' block statement}

  \begin{itemize}
    \item A block if-then statement is more flexible than the if statment
    \vfill\item It has the following form:\\
    \vfill\texttt{if (logical expression) then \\
	\hspace{0.5cm}execute statements  \\
	\hspace{0.5cm}more execute statements  \\
	end if \\}
    \item If the logical expression is \texttt{.true.} then the block is executed
  \end{itemize}
  \vfill
  \texttt{if (coremap(i,j,k) == 99999) then \\
    \hspace{0.5cm} beta = eval\_albedo(i,j,k) \\
    \hspace{0.5cm} D = beta(2)/beta(1) \\
    \hspace{0.5cm} Dhat = D/dx*(flux(i,j,k)) \\
    end if
    }
  
\end{frame}
%------------------------------------------------------------------------------
\begin{frame}{The 'if-else' block statement}

  \begin{itemize}
    \item A block if-then statement is more flexible than the if statment
    \vfill\item It has the following form:\\
    \vfill\texttt{if (logical expression) then \\
	\hspace{0.5cm}execute statements  \\
	else \\
	\hspace{0.5cm}execute statements  \\
	end if \\}
    \item If the logical expression is \texttt{.true.} then the first block is executed. Otherwise, the second block is omitted
  \end{itemize}
  \vfill
  \texttt{if (coremap(i,j,k) == 99999) then \\
      \hspace{0.5cm} beta = eval\_albedo(i,j,k) \\
    else \\
      \hspace{0.5cm} beta = 1.0 \\
    end if
    }
  
\end{frame}
%------------------------------------------------------------------------------
\begin{frame}{Including the 'else if'}
  \begin{itemize}
    \item You can use as many \texttt{else if} statements as you want
    \vfill\item There is only 1 \texttt{end if} per \texttt{if} (none for else if)
    \vfill\item All {else if} statements must be listed before the \texttt{else}
    \vfill\item They are checked in order until the first \texttt{.true.} is evaluated
    \vfill\item You don't have to have the \texttt{else} statement
  \end{itemize}
\end{frame}
%------------------------------------------------------------------------------
\begin{frame}{Logical Operators}
     Order of logical operation is used
  \begin{itemize}
    \vfill\item \texttt{()} - inside out, and function references
    \vfill\item \texttt{.not.} - left to right
    \vfill\item \texttt{.and.} - left to right
    \vfill\item \texttt{.or.} - left to right
    \vfill\item \texttt{==,/=,>,<,>=,<=} - left to right
  \end{itemize}
\end{frame}
%------------------------------------------------------------------------------
\begin{frame}[allowframebreaks]{Square root}
  \begin{scriptsize}
    \lstinputlisting{../examples/squareroot.f90}
  \end{scriptsize}
\end{frame}
%------------------------------------------------------------------------------
\begin{frame}{End of class/External Assignment}

  \begin{itemize}
    \item Write a program that solves the quadratic equation
    \begin{equation}
      \nonumber
       y = ax^{2} + bx + c 
    \end{equation}
    \vfill\item Have the user enter coefficients \emph{a}, \emph{b} and \emph{c}
    \vfill\item Depending on the sign of the discriminant you will have 3 rules:
    \begin{itemize}
      \vfill\item Imaginary, Real, or Double real root
    \end{itemize}
    \vfill\item Calculate appropriate number of solutions
    \vfill\item Print result to user
  \end{itemize}

\end{frame}
%------------------------------------------------------------------------------
\end{document}