%%%%%%%%%%%%   LaTeX Preamble %%%%%%%%%%%%%%

\documentclass{beamer}

% list all packages
\usepackage{amsmath}
\usepackage{hyperref}
\usepackage{latexsym}
\usepackage{color}
\usepackage{mycommands}

% slide theme
\usetheme{Berlin}
\usecolortheme{mit}

% Set Logo
\pgfdeclareimage[height=0.5cm]{mit-logo}{../../mit-logo.pdf}
\logo{\vspace{-0.25cm}\pgfuseimage{mit-logo}\hspace*{0.025cm}}

% Include Custom environments
% % \setbeamertemplate{blocks}[rounded]
\setbeamertemplate{blocks}[rounded][shadow=true]

% \setbeamertemplate{headline}[default]
\setbeamertemplate{navigation symbols}{}

% \beamerdefaultoverlayspecification{<+->}

% Set color for 'alert' text
\setbeamercolor{alerted text}{fg=blue}

%\setbeamercolor{section in toc}

% Modify some default font sizes
\setbeamerfont{itemize/enumerate body}{size=\normalfont}
\setbeamerfont{itemize/enumerate subbody}{size=\smaller, shape=\upshape}
\setbeamerfont{frametitle}{size=\large, series=\bfseries}


% \setbeamertemplate{bibliography entry title}{}
% \setbeamertemplate{bibliography entry location}{}
% \setbeamertemplate{bibliography entry note}{}
\setbeamertemplate{bibliography item}[text] 

% \setbeamertemplate{items}[ball]
% \setbeamertemplate{itemize subitem}[circle-symbol]
% \setbeamertemplate{background canvas}[vertical shading][bottom=mitgray!25,top=white]

% Use the shrink option to squeeze lots of text on a slide

% \frame[shrink]{

% …

% }

\colorlet{dark green}{green!50!black}

\newcommand{\packin}{\setlength\abovedisplayskip{2pt}\setlength\belowdisplayskip{2pt}}

%\newcommand{\numberInBox}[2][0.9]%
%	{\scalebox{#1}{{\tikz \draw (0,0) node[refbox] {\makebox[\totalheight]{#2}};}}}

%\newcommand{\enumref}[2][0.9] {\numberInBox[#1]{\ref{#2}}}


% Small arrow pointing down and hooking right
\newcommand{\drarrow}{\scalebox{1.5}{\reflectbox{\rotatebox[c]{180}{$\boldsymbol{\smash[b]{\Rsh}}$}}}}

\newenvironment{prettydescript}[1]
	{\begin{list}{}%
		{\renewcommand\makelabel[1]{\itshape\bfseries\color{mitred} ##1:\hfill}%
		\settowidth\labelwidth{\makelabel{#1}}%
		\setlength\leftmargin{\labelwidth}%
		\addtolength\leftmargin{\labelsep}}}%
	{\end{list}}

\newenvironment{customdescript}[1]
	{\begin{list}{}%
		{\renewcommand\makelabel[1]{\bfseries\color{mitred} ##1\hfill}%
		\settowidth\labelwidth{\makelabel{#1}}%
		\setlength\leftmargin{\labelwidth}%
		\addtolength\leftmargin{\labelsep}}}%
	{\end{list}}

\makeatletter

\newenvironment{customlist}[2]{
  \ifnum\@itemdepth >2\relax\@toodeep\else
      \advance\@itemdepth\@ne%
      \beamer@computepref\@itemdepth%
      \usebeamerfont{itemize/enumerate \beameritemnestingprefix body}%
      \usebeamercolor[fg]{itemize/enumerate \beameritemnestingprefix body}%
      \usebeamertemplate{itemize/enumerate \beameritemnestingprefix body begin}%
      \begin{list}
        {
            \usebeamertemplate{itemize \beameritemnestingprefix item}
        }
        { \leftmargin=#1 \itemindent=#2
            \def\makelabel##1{%
              {%  
                  \hss\llap{{%
                    \usebeamerfont*{itemize \beameritemnestingprefix item}%
                        \usebeamercolor[fg]{itemize \beameritemnestingprefix item}##1}}%
              }%  
            }%  
        }
  \fi
}
{
  \end{list}
  \usebeamertemplate{itemize/enumerate \beameritemnestingprefix body end}%
}
\makeatother

\newenvironment<>{varblock}[2][\textwidth]{%
  \setlength{\textwidth}{#1}
  \begin{actionenv}#3%
    \def\insertblocktitle{#2}%
    \par%
    \usebeamertemplate{block begin}}
  {\par%
    \usebeamertemplate{block end}%
  \end{actionenv}}

%% Notational commands:
\newcommand{\params}{\ensuremath{\xi}}
\newcommand{\vparms}{\ensuremath{\gvect{\params}}}
\renewcommand{\thefootnote}{\ensuremath{\fnsymbol{footnote}}}
\setcounter{footnote}{2}
\renewcommand{\thempfootnote}{\ensuremath{\fnsymbol{mpfootnote}}}
\newcommand{\newsubsection}[1]{\subsection{#1}\setcounter{subsection}{0}}


% Title Page
\title[Data Types and Intrinsic Functions]{Data Types and Intrinsic Functions}
\author[]{22.901 Introduction to Computer Programming for Nuclear Engineers}
\institute[\insertpagenumber]{}
\date{January 19, 2012} 

% -----------------------------------------------------------------------------
\begin{document}
% -----------------------------------------------------------------------------

% Inset title page
\frame{\titlepage}

% Outline slide
\begin{frame}{Outline}
  \begin{itemize}
   \item Intro to programming
  \end{itemize}
\end{frame}
%------------------------------------------------------------------------------
\begin{frame}{Ensuring Explicit Variable Declaration}

  \begin{itemize}
    
    \item Lecture 1 highlighted that Fortran has implicit type designation depending on the first letter
    \vfill\item In reality, we do not want to be restricted to this so we will \emph{declare} all varibles
    \vfill\item When declaring variables we give the type, attributes, comments and other options
    \vfill\item If we forget to declare a variable, Fortran will use the implicit naming scheme!
      \begin{itemize}
	\item What is the danger in this?
      \end{itemize}
    \vfill\item To ensure that no variable is declared implicitly we use \texttt{implicit none}
    \vfill\item This statement should be listed a the top of every routine!

  \end{itemize}

\end{frame}
%------------------------------------------------------------------------------
\begin{frame}{DataTypes}

  \begin{itemize}
    \item \textbf{integer} for exact whole numbers
    \vfill\item \textbf{real} for approximate, fractional numbers
    \vfill\item \textbf{complex} for complex, fraction numbers
    \vfill\item \textbf{logical} for boolean values \texttt{.true.} and \texttt{.false.}
    \vfill\item \textbf{character} for strings of characters
    \vfill\item \alert{Note:} strings are constructured from an \emph{array} of characters 
  \end{itemize}

\end{frame}
%------------------------------------------------------------------------------
\begin{frame}{When to use Integer vs. Real}

Integers::
  \begin{itemize}
    \item Always think to yourself is this variable always a whole number?
    \item Loop counts or loop limits
    \item An index into an array
    \item For error codes
  \end{itemize}

\vfill Reals::
  \begin{itemize}
    \item Are held as floating-point values 
    \item Can be defined with decimals and exponents
  \end{itemize}

\end{frame}
%------------------------------------------------------------------------------
\begin{frame}{How to declare a variable}

  \texttt{integer  :: i                   ! loop iteration counter for x} \\
  \texttt{real     :: flux                ! scalar for total neutron flux} \\
  \texttt{logical  :: finished = .false.  ! code completed?} \\
  \texttt{character(len=8) :: astr        ! a string} \\
\vfill 
 \begin{itemize}
    \item All of the different type of declarations are above
    \vfill\item You can set values right in the declarations
    \vfill\item The '::' symbol separates the type info from the variable
    \vfill\item \alert{Single vs. Double Precision}
      \begin{itemize}
	\item Default is 4 bytes (single precision) real $\equiv$ real(4)
	\item To get double precision use real(8), 8 bytes
      \end{itemize}
  \end{itemize}
\end{frame}
%------------------------------------------------------------------------------
\begin{frame}{Type Conversions}

\textbf{In any expression it is important to have the same variable type!}
\vfill
  \begin{itemize}
    \item\texttt{real(arg)} -- converts \texttt{arg} to a real(4)
    \vfill\item\texttt{int(arg)} -- converts \texttt{arg} to a integer(4)
    \vfill\item\texttt{dble(arg)} -- converts \texttt{arg} to a real(8).
    \vfill\item\texttt{ceiling(x)} -- smallest integer greater or equal to \texttt{x}
    \vfill\item\texttt{floor(x)} -- largest integer less or equal to \texttt{x}
    \vfill\item\texttt{nint(x)} -- nearest integer to \texttt{x}
  \end{itemize}

\end{frame}
%------------------------------------------------------------------------------
\begin{frame}
\begin{center}
\Huge{Example 1}
\end{center}
\end{frame}
%------------------------------------------------------------------------------
\begin{frame}{Fortran Intrinsic Functions}

Fortran intrinsic functions are built-in routines that you can call
\vfill
\begin{itemize}
  \item \texttt{y = sqrt(x)}
  \vfill\item \texttt{pi = 4.0*atan(1.0)}
  \vfill\item \texttt{z = exp(3.0*y)}
  \vfill\item \texttt{u = log(x)} natural log use \texttt{log10} for base 10
  \vfill\item \texttt{sin(x),cos(x),tan(x) ...} {\color{red}angles are in rads}
  \vfill\item \texttt{max(x,y), min(x,y)}
\end{itemize}
\vfill
There are so many, please search internet if you are looking for something specific!
\end{frame}
%------------------------------------------------------------------------------
\begin{frame}{Character Type}
  \begin{itemize}
    \item Used when strings of characters are required
    \vfill\item Fortran's basic type is a fixed-length string
    \vfill\item Character constants are quoted strings
    \begin{itemize}
      \item \texttt{print *, 'This is a title'}
      \item \texttt{print *, ``And so is this''}
    \end{itemize}
    \vfill\item The characters between quotes are the value
  \end{itemize}
\end{frame}
%------------------------------------------------------------------------------
\begin{frame}{Character Variables}
\texttt{character :: answer, marital\_status} \\
\texttt{character(len=10) :: name,dept,faculty} \\
\texttt{character(len=32) :: address} \\
\vfill
\texttt{answer} and \texttt{marital\_status} are each of length \textbf{1}. \\
They hold precisely one character each such are 'Y' or 'n'. \\
\vfill
\texttt{name}, \texttt{dept} and \texttt{faculty} are of length \textbf{10}. \\
\texttt{address} is of length \textbf{32}. \\
\vfill
{\color{red}If you do not use the full 'len' the rest is whitespace!}
\end{frame}
%------------------------------------------------------------------------------
\begin{frame}{Character Concatenation}
  
  Strings may be joined using the \texttt{//} operator \\
\vfill
  \texttt{character(len=6) :: identity,a,b,z \\
    identity = 'TH' // 'OMAS' \\
    a = 'TH' \\
    b = 'OMAS' \\
    z = a // b \\
    print *,identity \\
    print *,z \\
  }
\vfill
What is the result?
\vfill
// does not remove trailing spaces!
\end{frame}
%------------------------------------------------------------------------------
\begin{frame}{Character Intrinsics}
\texttt{len(c) -- the storage length of c\\
  trim(c) -- remove trailing blanks\\
  adjustl(c) -- remove leading blanks\\
  index(str,sub) -- position of a substring in string
  repeat(str,num) -- copy a str num times and join
}
\vfill Again there are many intrinsics to search internet!
\vfill
\texttt{z = trim(a)//trim(b)} \\
\texttt{print *,z} \\
\vfill
This will give us THOMAS!
\end{frame}
%------------------------------------------------------------------------------
\begin{frame}{Parameters or Constants}
  \begin{itemize}
    \item \texttt{parameter} is an optional attribute for a variable
    \vfill\item parameters cannot be changed during the execution of the code
    \vfill\item the actual values of the parameters are specific and substituted in a compile time
    \vfill\item this can reduce errors for variables that should not change
  \end{itemize}
\vfill
\texttt{real(8), parameter :: pi = 3.14159\_8 (or 3.14159D0) \\
	integer, parameter :: maxlen = 100}
\end{frame}
%------------------------------------------------------------------------------
\begin{frame}
\begin{center}
\Huge{Example 2}
\end{center}
\end{frame}
%------------------------------------------------------------------------------
\begin{frame}{End of Class/External Assignment}
\center\Large\textbf{Law of Cosines}
\normalsize
\begin{itemize}
  \vfill\item Declare all variables
  \vfill\item Get user input for 2 sides and an angle between (read in degrees)
  \vfill\item Also from user get name of the third side (The length (sidename) is)
  \vfill\item Do out the math (be careful!)
  \vfill\item Print results to user
\end{itemize}
\end{frame}
%------------------------------------------------------------------------------
\end{document}