%%%%%%%%%%%%   LaTeX Preamble %%%%%%%%%%%%%%

\documentclass{beamer}

% list all packages
\usepackage{amsmath}
\usepackage{hyperref}
\usepackage{latexsym}
\usepackage{color}
\usepackage{mycommands}
\usepackage{listings}
\lstset{language=Fortran,
        keywordstyle=\color{blue}\textbf,
	commentstyle=\color[rgb]{0.133,0.545,0.133},
	stringstyle=\color[rgb]{0.627,0.126,0.941},
	breaklines=true,
	showstringspaces=false,
	frame=trBL
       }

% slide theme
\usetheme{Berlin}
\usecolortheme{mit}

% Set Logo
\pgfdeclareimage[height=0.5cm]{mit-logo}{../../mit-logo.pdf}
\logo{\vspace{-0.25cm}\pgfuseimage{mit-logo}\hspace*{0.025cm}}

% Include Custom environments
% % \setbeamertemplate{blocks}[rounded]
\setbeamertemplate{blocks}[rounded][shadow=true]

% \setbeamertemplate{headline}[default]
\setbeamertemplate{navigation symbols}{}

% \beamerdefaultoverlayspecification{<+->}

% Set color for 'alert' text
\setbeamercolor{alerted text}{fg=blue}

%\setbeamercolor{section in toc}

% Modify some default font sizes
\setbeamerfont{itemize/enumerate body}{size=\normalfont}
\setbeamerfont{itemize/enumerate subbody}{size=\smaller, shape=\upshape}
\setbeamerfont{frametitle}{size=\large, series=\bfseries}


% \setbeamertemplate{bibliography entry title}{}
% \setbeamertemplate{bibliography entry location}{}
% \setbeamertemplate{bibliography entry note}{}
\setbeamertemplate{bibliography item}[text] 

% \setbeamertemplate{items}[ball]
% \setbeamertemplate{itemize subitem}[circle-symbol]
% \setbeamertemplate{background canvas}[vertical shading][bottom=mitgray!25,top=white]

% Use the shrink option to squeeze lots of text on a slide

% \frame[shrink]{

% …

% }

\colorlet{dark green}{green!50!black}

\newcommand{\packin}{\setlength\abovedisplayskip{2pt}\setlength\belowdisplayskip{2pt}}

%\newcommand{\numberInBox}[2][0.9]%
%	{\scalebox{#1}{{\tikz \draw (0,0) node[refbox] {\makebox[\totalheight]{#2}};}}}

%\newcommand{\enumref}[2][0.9] {\numberInBox[#1]{\ref{#2}}}


% Small arrow pointing down and hooking right
\newcommand{\drarrow}{\scalebox{1.5}{\reflectbox{\rotatebox[c]{180}{$\boldsymbol{\smash[b]{\Rsh}}$}}}}

\newenvironment{prettydescript}[1]
	{\begin{list}{}%
		{\renewcommand\makelabel[1]{\itshape\bfseries\color{mitred} ##1:\hfill}%
		\settowidth\labelwidth{\makelabel{#1}}%
		\setlength\leftmargin{\labelwidth}%
		\addtolength\leftmargin{\labelsep}}}%
	{\end{list}}

\newenvironment{customdescript}[1]
	{\begin{list}{}%
		{\renewcommand\makelabel[1]{\bfseries\color{mitred} ##1\hfill}%
		\settowidth\labelwidth{\makelabel{#1}}%
		\setlength\leftmargin{\labelwidth}%
		\addtolength\leftmargin{\labelsep}}}%
	{\end{list}}

\makeatletter

\newenvironment{customlist}[2]{
  \ifnum\@itemdepth >2\relax\@toodeep\else
      \advance\@itemdepth\@ne%
      \beamer@computepref\@itemdepth%
      \usebeamerfont{itemize/enumerate \beameritemnestingprefix body}%
      \usebeamercolor[fg]{itemize/enumerate \beameritemnestingprefix body}%
      \usebeamertemplate{itemize/enumerate \beameritemnestingprefix body begin}%
      \begin{list}
        {
            \usebeamertemplate{itemize \beameritemnestingprefix item}
        }
        { \leftmargin=#1 \itemindent=#2
            \def\makelabel##1{%
              {%  
                  \hss\llap{{%
                    \usebeamerfont*{itemize \beameritemnestingprefix item}%
                        \usebeamercolor[fg]{itemize \beameritemnestingprefix item}##1}}%
              }%  
            }%  
        }
  \fi
}
{
  \end{list}
  \usebeamertemplate{itemize/enumerate \beameritemnestingprefix body end}%
}
\makeatother

\newenvironment<>{varblock}[2][\textwidth]{%
  \setlength{\textwidth}{#1}
  \begin{actionenv}#3%
    \def\insertblocktitle{#2}%
    \par%
    \usebeamertemplate{block begin}}
  {\par%
    \usebeamertemplate{block end}%
  \end{actionenv}}

%% Notational commands:
\newcommand{\params}{\ensuremath{\xi}}
\newcommand{\vparms}{\ensuremath{\gvect{\params}}}
\renewcommand{\thefootnote}{\ensuremath{\fnsymbol{footnote}}}
\setcounter{footnote}{2}
\renewcommand{\thempfootnote}{\ensuremath{\fnsymbol{mpfootnote}}}
\newcommand{\newsubsection}[1]{\subsection{#1}\setcounter{subsection}{0}}


% Title Page
\title[]{Formatted I/O, Derived Types and Makefiles}
\author[]{22.901 Introduction to Computer Programming for Nuclear Engineers}
\institute[\insertpagenumber]{}
\date{January 26, 2012} 

% -----------------------------------------------------------------------------
\begin{document}
% -----------------------------------------------------------------------------

% Inset title page
\frame{\titlepage}

% Outline slide
\begin{frame}{Outline}
  \begin{itemize}
    \item Formatted I/O
    \vfill\item Derived Types
    \vfill\item Makefiles
  \end{itemize}
\end{frame}
%------------------------------------------------------------------------------
\begin{frame}{What Have we Done so Far?}

  \begin{itemize}
   \item \texttt{print *,} - used for printing to screen
   \item \texttt{read *,} - used for reading from screen
  \end{itemize}
  \vfill
  These are historical, in modern Fortran we use:
  \begin{itemize}
    \item \texttt{write(*,*)<stuff>} - write to screen
    \item \texttt{read(*,*)<stuff>} - read from screen
  \end{itemize}
  \vfill
  Really, this is shorthand notation for: \\
  \begin{center}
    \texttt{write(unit=*,fmt=*)}
  \end{center}
  \vfill
  \begin{itemize}
    \item \texttt{unit=*} stdin(unit=6) or stdout=(unit=5)
    \item \texttt{fmt=*} list directed input/output
  \end{itemize}

\end{frame}
%------------------------------------------------------------------------------
\begin{frame}{Using Units}

  \begin{itemize}
    \item A unit is an integer identifying a connection to a file
    \vfill\item Generally use values bewteen 10 and 99
    \vfill\item \texttt{unit=} can be omitted if the unit is first in the write list
  \end{itemize}
  \vfill
  \begin{center}
    \texttt{write(10,*) 'Hello'}
  \end{center}
  \begin{itemize}
    \vfill\item This will write the string Hello to unit 10
    \vfill\item By default, unit 10 directs to the file 'fort.10'
  \end{itemize}

\end{frame}
%------------------------------------------------------------------------------
\begin{frame}{Specifying a Format}

  \begin{itemize}
    \item \texttt{fmt=} can be omitted if the format is second and the first is unit
    \vfill\item \texttt{fmt=*} indicates list directed I/O
    \vfill\item \texttt{fmt=<fmt>} indicates a fixed format
    \vfill\item If \texttt{<fmt>} is a number, a format card is needed
  \end{itemize}

\end{frame}
%------------------------------------------------------------------------------
\begin{frame}{Implied DO Loops}

  \begin{itemize}
    \item There is an alternative to array expressions
    \vfill\item Often more convinient
    \vfill\item The format: \texttt{ (<list>,<indexed loop control>) }
  \end{itemize}
  \vfill
  For example:
  \begin{center}
    \texttt{((A(i,j),j=1,3),B(i),i=6,2,-2)} \\
  \end{center}
  \vfill
  This prints: \\
  \texttt{A(6,1),A(6,2),A(6,3),B(6),A(4,1),A(4,2),A(4,3),B(4),...}

\end{frame}
%------------------------------------------------------------------------------
\begin{frame}{Integer Descriptor}

  \begin{itemize}
    \item \texttt{In} displays an integer
    \vfill\item Integers are printed \textbf{right-justified}
    \vfill\item \texttt{In.m} displays at least \texttt{m} digits
    \vfill\item A '-' sign will take up one spot in the field
  \end{itemize}
  \vfill
  For example:
  \begin{center}
    \texttt{write(*,'(I7)') 123} will give  '\texttt{\^}\texttt{\^}\texttt{\^}\texttt{\^}123' \\
    \texttt{write(*,'(I7,5)') 123} will give '\texttt{\^}\texttt{\^}00123'
  \end{center}
  \vfill
  \alert{Note:} The character \texttt{\^} does not actually appear, it is just used to denote a space in this presentation.
\end{frame}
%------------------------------------------------------------------------------
\begin{frame}{Fixed-Formal Real}

  \begin{itemize}
    \item \texttt{Fn.m} displays to \texttt{m} decimal places
    \vfill\item Printed \textbf{right-justified} in a field of width \texttt{n}
    \vfill\item A '-' and the decimal '\texttt{.}' take up one spot in the filed
    \vfill\item Numbers should be correctly rounded
  \end{itemize}
  \vfill
  For example:
  \begin{center}
    \texttt{write(*,'(F9.3)') 1.23} will give '\texttt{\^}\texttt{\^}\texttt{\^}\texttt{\^}1.230'
    \texttt{write(*,'(F9.5)') 0.123e-4} will give '\texttt{\^}\texttt{\^}0.00001'
  \end{center}
  \vfill
  \alert{Note:} The character \texttt{\^} does not actually appear, it is just used to denote a space in this presentation.

\end{frame}
%------------------------------------------------------------------------------
\begin{frame}{Field Overflow and Zero Widths}

  \begin{itemize}
    \item A field overflow happens when you try to fit a value in a field that is too small
    \vfill\item the whole field is then replaced by asterisks
    \vfill\item putting 12345 into I4 gives ****
    \vfill\item for integer and fixed-format reals using a width of 0 will get rid of leading spaces
  \end{itemize}
  \vfill
  For example: \\
    \texttt{write(*,'(I7)') 123} will give  '\texttt{\^}\texttt{\^}\texttt{\^}\texttt{\^}123' \\
    \texttt{write(*,'(I0)') 123} will give  '123' \\
    \texttt{write(*,'(``/'',I0,``/'',F0.3)')} 12345,987.654321 will give '/12345/987.654'

\end{frame}
%------------------------------------------------------------------------------
\begin{frame}{Exponential Format}

  \begin{itemize}
    \item Use the \texttt{ESn.m} for \texttt{m} decimal places
    \vfill\item \texttt{n} is the total field with including '-' and '.' and exponential part (usually 4 spots)
    \vfill\item General Rule: $n \ge m+7$
  \end{itemize}
  \vfill
  For example: \\
  \vfill
  \texttt{write(*,'(ES12.5)') 233323.2324234} will give 2.33323E+05
  \vfill
\end{frame}
%------------------------------------------------------------------------------
\begin{frame}{Character Descriptor}

  \begin{itemize}
    \item \texttt{An} displays a field width of \texttt{n}
    \vfill\item \texttt{A} uses the exact width of the output item
    \vfill\item If the output field is too small, the left most characters are used
    \vfill\item If the output field is too large, \textbf{right-justified}
  \end{itemize}
  \vfill
  For example: \\
  \texttt{write(*,'(A3)') 'abcdefgh'} will give \texttt{abc}

\end{frame}
%------------------------------------------------------------------------------
\begin{frame}{Other Descriptors}

  \begin{itemize}
    \item \texttt{Ln} displays either T or F
    \vfill\item \texttt{Gn} or \texttt{Gn.m} is a generalized descriptor
    \vfill\item You can do string contants by surrouding in quotes
    \vfill\item Spacing descriptor - \texttt{nX}. 4X puts 4 spaces
    \vfill\item Newline descriptor - just do a single /
    \vfill\item Tab descriptor - \texttt{Tn} where \texttt{n} is the column number to tab to 
  \end{itemize}

\end{frame}
%------------------------------------------------------------------------------
\begin{frame}{Opening a file}

  \begin{itemize}
    \item Use the \text{open} command
    \vfill
    \begin{center}
      \texttt{open(unit=11,file='fred')}
    \end{center}
    \vfill\item Use unit 11 now in all write or read statements
    \vfill\item Use numbers 10-99
    \vfill\item To close a file use close(unit\#)
  \end{itemize}

\end{frame}
%------------------------------------------------------------------------------
\begin{frame}[allowframebreaks]{Matrix Multiply}
\begin{scriptsize}
  \lstinputlisting{../examples/matmultiply.f90}
\end{scriptsize}
\end{frame}
%------------------------------------------------------------------------------
\begin{frame}{What is a derived type}

  \begin{itemize}
    \item Basically is it a grouping of variables
    \vfill\item This is sometimes referred to as a structure
    \vfill\item These are very useful in organizing data
    \vfill\item Think of the underlying data as attributes
  \end{itemize}
  \vfill
  For example: \\
  \texttt{type :: particle} \\
    \hspace{0.1cm} \texttt{real :: xcoord ! particle location} \\
    \hspace{0.1cm} \texttt{real :: energy ! particle energy} \\
    \hspace{0.1cm} \texttt{real :: angle  ! particle angle} \\
    \hspace{0.1cm} \texttt{real :: weight ! statistical weight} \\
    \hspace{0.1cm} \texttt{logical :: alive ! particle alive?} \\
  \texttt{end type particle}
\end{frame}
%------------------------------------------------------------------------------
\begin{frame}{Component Selection}

  \begin{itemize}
    \item The selector '\%' is used for this
    \vfill\item It is then followed by a component of the derived type
  \end{itemize}
  \vfill
  For example: \\
  \texttt{type(particle) :: neutron} \\
  \texttt{neutron\%energy = 1.e6} \\
  \texttt{neutron\%xcoord = start\_neutron()}
  \vfill
\end{frame}
%------------------------------------------------------------------------------
\begin{frame}{Array of Types}

  \begin{itemize}
    \item You can have an array of types
  \end{itemize}
  \vfill
  For example:
  \texttt{type :: tally} \\
    \hspace{0.1cm} \texttt{real :: s1} \\
    \hspace{0.1cm} \texttt{real :: s2} \\
    \hspace{0.1cm} \texttt{real :: mean} \\
    \hspace{0.1cm} \texttt{real :: var} \\
  \texttt{end type tally} \\
  \texttt{type(tally), allocatable :: tallies(:)} \\
  \texttt{allocate(tallies(n\_slabs))} \\
  \texttt{tallies(1)\%mean = 0.0}
  \vfill

\end{frame}
%------------------------------------------------------------------------------
\begin{frame}{Assignment}

  \begin{itemize}
    \item You can assign complete derived types
    \vfill\item Or simply just components of types
    \vfill\item \alert{Note:} They have to be of the \textbf{same} type
  \end{itemize}
  \vfill
  \texttt{type(bike) :: mine, yours} \\
  \texttt{yours = mine} \\
  \texttt{mine\%front = yours\%back}

\end{frame}
%------------------------------------------------------------------------------
\begin{frame}{Makefiles}

  \begin{itemize}
    \item You need to set up rules to compile the modules
    \vfill\item You need to add dependencies to ensure they are rebuilt
  \end{itemize}
  \vfill
  Here is a rule to compile objects: \\
  \texttt{program: <objects>} \\
  \texttt{<tab> \$(FC) \$(FFLAGS) \$(LDFLAGS) -o \$@ \$\^}
  \begin{itemize}
    \vfill\item \$(var) just represents a variable that was set already in the makefile
    \vfill\item \texttt{<objects>} is a list of objects in order of dependencies
  \end{itemize}
  \begin{center}
    \alert{Please search online for more advanced features}
  \end{center}

\end{frame}
%------------------------------------------------------------------------------
\begin{frame}{End of Class Assignment}

  \begin{itemize}
    \item Solve for pi via Monte Carlo
    \vfill\item Read in number of particles to simulate
    \vfill\item Set 1cm bounds for 1/4 dartboard
    \vfill\item Randomly throw a dart with \texttt{rand} intrinsic function
    \vfill\item Record how many darts are within quarter circle
    \vfill\item Set up ratio between 1/4 circle and square areas and darts
    \vfill\item Solve for the mean of pi and report that and variance
  \end{itemize}

\end{frame}
%------------------------------------------------------------------------------
\end{document}