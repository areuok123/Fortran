%%%%%%%%%%%%   LaTeX Preamble %%%%%%%%%%%%%%

\documentclass{beamer}

% list all packages
\usepackage{amsmath}
\usepackage{hyperref}
\usepackage{latexsym}
\usepackage{color}
\usepackage{mycommands}
\usepackage{listings}
\lstset{language=Fortran,
        keywordstyle=\color{blue}\textbf,
	commentstyle=\color[rgb]{0.133,0.545,0.133},
	stringstyle=\color[rgb]{0.627,0.126,0.941},
	breaklines=true,
	showstringspaces=false,
	frame=trBL
       }

% slide theme
\usetheme{Berlin}
\usecolortheme{mit}

% Set Logo
\pgfdeclareimage[height=0.5cm]{mit-logo}{../mit-logo.pdf}
\logo{\vspace{-0.25cm}\pgfuseimage{mit-logo}\hspace*{0.025cm}}

% Include Custom environments
% % \setbeamertemplate{blocks}[rounded]
\setbeamertemplate{blocks}[rounded][shadow=true]

% \setbeamertemplate{headline}[default]
\setbeamertemplate{navigation symbols}{}

% \beamerdefaultoverlayspecification{<+->}

% Set color for 'alert' text
\setbeamercolor{alerted text}{fg=blue}

%\setbeamercolor{section in toc}

% Modify some default font sizes
\setbeamerfont{itemize/enumerate body}{size=\normalfont}
\setbeamerfont{itemize/enumerate subbody}{size=\smaller, shape=\upshape}
\setbeamerfont{frametitle}{size=\large, series=\bfseries}


% \setbeamertemplate{bibliography entry title}{}
% \setbeamertemplate{bibliography entry location}{}
% \setbeamertemplate{bibliography entry note}{}
\setbeamertemplate{bibliography item}[text] 

% \setbeamertemplate{items}[ball]
% \setbeamertemplate{itemize subitem}[circle-symbol]
% \setbeamertemplate{background canvas}[vertical shading][bottom=mitgray!25,top=white]

% Use the shrink option to squeeze lots of text on a slide

% \frame[shrink]{

% …

% }

\colorlet{dark green}{green!50!black}

\newcommand{\packin}{\setlength\abovedisplayskip{2pt}\setlength\belowdisplayskip{2pt}}

%\newcommand{\numberInBox}[2][0.9]%
%	{\scalebox{#1}{{\tikz \draw (0,0) node[refbox] {\makebox[\totalheight]{#2}};}}}

%\newcommand{\enumref}[2][0.9] {\numberInBox[#1]{\ref{#2}}}


% Small arrow pointing down and hooking right
\newcommand{\drarrow}{\scalebox{1.5}{\reflectbox{\rotatebox[c]{180}{$\boldsymbol{\smash[b]{\Rsh}}$}}}}

\newenvironment{prettydescript}[1]
	{\begin{list}{}%
		{\renewcommand\makelabel[1]{\itshape\bfseries\color{mitred} ##1:\hfill}%
		\settowidth\labelwidth{\makelabel{#1}}%
		\setlength\leftmargin{\labelwidth}%
		\addtolength\leftmargin{\labelsep}}}%
	{\end{list}}

\newenvironment{customdescript}[1]
	{\begin{list}{}%
		{\renewcommand\makelabel[1]{\bfseries\color{mitred} ##1\hfill}%
		\settowidth\labelwidth{\makelabel{#1}}%
		\setlength\leftmargin{\labelwidth}%
		\addtolength\leftmargin{\labelsep}}}%
	{\end{list}}

\makeatletter

\newenvironment{customlist}[2]{
  \ifnum\@itemdepth >2\relax\@toodeep\else
      \advance\@itemdepth\@ne%
      \beamer@computepref\@itemdepth%
      \usebeamerfont{itemize/enumerate \beameritemnestingprefix body}%
      \usebeamercolor[fg]{itemize/enumerate \beameritemnestingprefix body}%
      \usebeamertemplate{itemize/enumerate \beameritemnestingprefix body begin}%
      \begin{list}
        {
            \usebeamertemplate{itemize \beameritemnestingprefix item}
        }
        { \leftmargin=#1 \itemindent=#2
            \def\makelabel##1{%
              {%  
                  \hss\llap{{%
                    \usebeamerfont*{itemize \beameritemnestingprefix item}%
                        \usebeamercolor[fg]{itemize \beameritemnestingprefix item}##1}}%
              }%  
            }%  
        }
  \fi
}
{
  \end{list}
  \usebeamertemplate{itemize/enumerate \beameritemnestingprefix body end}%
}
\makeatother

\newenvironment<>{varblock}[2][\textwidth]{%
  \setlength{\textwidth}{#1}
  \begin{actionenv}#3%
    \def\insertblocktitle{#2}%
    \par%
    \usebeamertemplate{block begin}}
  {\par%
    \usebeamertemplate{block end}%
  \end{actionenv}}

%% Notational commands:
\newcommand{\params}{\ensuremath{\xi}}
\newcommand{\vparms}{\ensuremath{\gvect{\params}}}
\renewcommand{\thefootnote}{\ensuremath{\fnsymbol{footnote}}}
\setcounter{footnote}{2}
\renewcommand{\thempfootnote}{\ensuremath{\fnsymbol{mpfootnote}}}
\newcommand{\newsubsection}[1]{\subsection{#1}\setcounter{subsection}{0}}


% Title Page
\title[Introduction to Computer Programming]{Introduction to Computer Programming}
\author[]{22.901 Introduction to Computer Programming for Nuclear Engineers}
\institute[\insertpagenumber]{}
\date{January 17, 2012} 

% -----------------------------------------------------------------------------
\begin{document}
% -----------------------------------------------------------------------------

% Inset title page
\frame{\titlepage}

% Outline slide
\begin{frame}{Outline}
  \begin{itemize}
   \item Intro to programming
  \end{itemize}
\end{frame}
%------------------------------------------------------------------------------
\begin{frame}{Programming Languages - Nuclear Perpsective}
\begin{itemize}

  \item Numerical Programming Languages
  \begin{itemize}
      \item Fortran, C, C++, etc.
      \item compiled languages
      \item very fast, low-level programming
  \end{itemize}

  \item Scripting Languages
    \begin{itemize}
      \item Python, Perl, Visual Basic,etc.
      \item interpreted languges
      \item good for data/file manipulation
      \item not as fast, high-level programming
    \end{itemize}

  \item Developmental Languages
    \begin{itemize}
      \item MATLAB, Python, etc.
      \item great enviorment for algorithm development
      \item excellent post-processing capability
      \item recommend for HWs and projects
    \end{itemize}

\end{itemize}
\end{frame}
%------------------------------------------------------------------------------
\begin{frame}{Coding Jargon}

\begin{itemize} 

  \item Source Code - an ASCII text file development by the programmer with code in it
  \vfill\item Compiler - the program that converts the source code to machine code
  \begin{itemize}
   \item C compilers - gcc, icc ...
   \item C++ compilers - g++ ...
   \item Fortran compilers - f77, g77, f90, f95, gfortran, ifort ...
  \end{itemize}
  \vfill\item Program - the compiled source code

\end{itemize}
\end{frame}
%------------------------------------------------------------------------------
\begin{frame}{Fortran Programming Language}

  \begin{itemize}
    \item In this course we will focus of Fortran
    \item Many nuclear engineering codes are \emph{still} developed in Fortran!

    \vfill\item Classical Fortran
      \begin{itemize}
	\item FORTRAN II-IV and Fortran 66 (1958-1966)
	\item FORTRAN 77 - standard programming language (still seen today!)
      \end{itemize}

    \vfill\item \alert{Modern Fortran}
    \begin{itemize}
      \item Fortran 90,95,2003,2008
      \item free format source code, structures, dynamic memory allocation
    \end{itemize}

  \end{itemize}
\end{frame}
%------------------------------------------------------------------------------
\begin{frame}{How can I use Fortran?}
\begin{itemize}
 
  \item Windows - Download Cygwin and get the gfortran compilers
  \vfill\item Mac/Linux - Download gfortran compilers
  \vfill\item In 22.901 we will SSH to the department's linux cluster
  \begin{itemize}
    \item Windows - download SecureFX/SecureCRT from MIT IST site
    \item Mac/Linux - SSH right from the terminal
  \end{itemize}
  \vfill\item To Login the following info is needed for SecureCRT
  \begin{itemize}
    \item hostname: \texttt{cheezit.mit.edu}
    \item username: \texttt{fortran12}
    \item passowrd: \texttt{22.901IAP2012}
    \item Mac/Linux: \texttt{ssh fortran12@cheezit.mit.edu}
  \end{itemize}
  \vfill\item see Stellar document about navigating a Linux shell
\end{itemize}
\end{frame}
%------------------------------------------------------------------------------
\begin{frame}{How to Compile Fortran Code with \texttt{gfortran}}
  \begin{itemize}
    \item Compile and Link \\
      \texttt{gfortran -o myCode myCode.f90}
    \vfill\item Compile only\\
      \texttt{gfortran -o myCode.o -c myCode.f90}
    \vfill\item Link only\\
      \texttt{gfortran -o myCode myCode.o}
    \vfill\item To include external files use \texttt{-I<path>}
    \vfill\item To link external libraries use \texttt{-L<path>}
  \end{itemize}
\end{frame}
%------------------------------------------------------------------------------
\begin{frame}{The Simplest Fortran Code}
\vfill
  \lstinputlisting{../examples/simplest.f90}
\vfill
\end{frame}
%------------------------------------------------------------------------------
\begin{frame}{Interacting with the User}
\begin{itemize}
  \item You may want to see what your code does!
  \vfill\item Use \texttt{print *,'str1',var1,var2,var3}
    \begin{itemize}
      \item This will print to the terminal screen (STDOUT) on one line 
      \item There are more advanced formatting methods (save for later)
    \end{itemize}
  \vfill\item To read something in from the user, use \texttt{read *,var1,var2}
  \vfill\item So now as a right of passage, print ``Hello World!'' to the screen
\end{itemize}
\end{frame}
%------------------------------------------------------------------------------
\begin{frame}{Hello World! Code}
\vfill
  \lstinputlisting{../examples/helloworld.f90}
\vfill
\end{frame}
%------------------------------------------------------------------------------
\begin{frame}{Code Structure}
A line can be up to 132 characters \\
A comment is from \alert{!} to the end of line  \\
You can have blank lines  \\
To continue a line place a \texttt{\&} at the end and begin the next line with \texttt{\&} \\
\vfill
The code structure ::
\begin{enumerate}
  \item start of a program, \texttt{program name1}
  \vfill\item declaration of all variables
  \vfill\item all of the execution statements
  \vfill\item end of program, \texttt{end program name1}
\end{enumerate}

\end{frame}
%------------------------------------------------------------------------------
\begin{frame}{Variables}
\begin{itemize}
  \item An ASCII alphanumeric word that represents a space in memory
  \item In Fortran the first character must be a letter (not case sensitive!!)
  \item By default variables beginning with I-L are integers, all else are reals (floats)
  \item This is the implicit naming structure of Fortran, later we will declare explicitly
  \item Some examples (Real or Integer ...)::
    \begin{itemize}
      \item \texttt{HYP} - ??
      \item \texttt{K9} - ??
      \item \texttt{pi} - ??
      \item \texttt{9X\_Y} - ??
    \end{itemize}
\end{itemize}
\end{frame}
%------------------------------------------------------------------------------
\begin{frame}{Simple Math}
\begin{itemize}
  \item Order of Operation Applies!
  \vfill\item Parentheses - ()
  \item Exponent - $**$
   \vfill \begin{itemize}
      \item if exponent is a whole number, list as integer (e.g. 2 not 2.0)
    \end{itemize}
  \vfill\item Multiplication/Division - *,/
  \vfill\item Addition/Subtraction - +,-
  \vfill\item Mathematical expressions should be of the same type (integer,real,etc.)
    \begin{itemize}
      \item $a=2$, $b=3$, $c=5$, $d=4$
      \item $e = (d + c)**a/b + d$ ???
    \end{itemize}
  \vfill\item Modulo - \texttt{a = mod(x,y)}
\end{itemize}
\end{frame}
%------------------------------------------------------------------------------
\begin{frame}{Integer vs. Real Division}
  \begin{itemize}
    \item Although not commonly used, division can be performed with integers
    \vfill\item The two numbers are divided and the decimal is truncated
    \vfill\item For example 5/3 = 1
    \vfill\item Division with Real variable types will give the correct answer
    \vfill\item For example 5/3 = 1.666666...
  \end{itemize}
\end{frame}
%------------------------------------------------------------------------------
\begin{frame}{End of Class/External Assignment}
Write a code to do the following:
\begin{enumerate}
  \item Ask user to enter any whole number 1-9
  \vfill\item Read in the number (real or integer??)
  \vfill\item Multiply the number by 9
  \vfill\item Add both digits together (hint: use Modulo)
  \vfill\item Subtract by 5
  \vfill\item Print out result \\ \vfill
\center{Try different numbers, notice anything??}
\end{enumerate}
\end{frame}
%------------------------------------------------------------------------------
\end{document}
