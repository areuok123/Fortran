%%%%%%%%%%%%   LaTeX Preamble %%%%%%%%%%%%%%

\documentclass{beamer}

% list all packages
\usepackage{amsmath}
\usepackage{hyperref}
\usepackage{latexsym}
\usepackage{color}
\usepackage{mycommands}
\usepackage{listings}
\lstset{language=Fortran,
        keywordstyle=\color{blue}\textbf,
	commentstyle=\color[rgb]{0.133,0.545,0.133},
	stringstyle=\color[rgb]{0.627,0.126,0.941},
	breaklines=true,
	showstringspaces=false,
	frame=trBL
       }

% slide theme
\usetheme{Berlin}
\usecolortheme{mit}

% Set Logo
\pgfdeclareimage[height=0.5cm]{mit-logo}{../../mit-logo.pdf}
\logo{\vspace{-0.25cm}\pgfuseimage{mit-logo}\hspace*{0.025cm}}

% Include Custom environments
% % \setbeamertemplate{blocks}[rounded]
\setbeamertemplate{blocks}[rounded][shadow=true]

% \setbeamertemplate{headline}[default]
\setbeamertemplate{navigation symbols}{}

% \beamerdefaultoverlayspecification{<+->}

% Set color for 'alert' text
\setbeamercolor{alerted text}{fg=blue}

%\setbeamercolor{section in toc}

% Modify some default font sizes
\setbeamerfont{itemize/enumerate body}{size=\normalfont}
\setbeamerfont{itemize/enumerate subbody}{size=\smaller, shape=\upshape}
\setbeamerfont{frametitle}{size=\large, series=\bfseries}


% \setbeamertemplate{bibliography entry title}{}
% \setbeamertemplate{bibliography entry location}{}
% \setbeamertemplate{bibliography entry note}{}
\setbeamertemplate{bibliography item}[text] 

% \setbeamertemplate{items}[ball]
% \setbeamertemplate{itemize subitem}[circle-symbol]
% \setbeamertemplate{background canvas}[vertical shading][bottom=mitgray!25,top=white]

% Use the shrink option to squeeze lots of text on a slide

% \frame[shrink]{

% …

% }

\colorlet{dark green}{green!50!black}

\newcommand{\packin}{\setlength\abovedisplayskip{2pt}\setlength\belowdisplayskip{2pt}}

%\newcommand{\numberInBox}[2][0.9]%
%	{\scalebox{#1}{{\tikz \draw (0,0) node[refbox] {\makebox[\totalheight]{#2}};}}}

%\newcommand{\enumref}[2][0.9] {\numberInBox[#1]{\ref{#2}}}


% Small arrow pointing down and hooking right
\newcommand{\drarrow}{\scalebox{1.5}{\reflectbox{\rotatebox[c]{180}{$\boldsymbol{\smash[b]{\Rsh}}$}}}}

\newenvironment{prettydescript}[1]
	{\begin{list}{}%
		{\renewcommand\makelabel[1]{\itshape\bfseries\color{mitred} ##1:\hfill}%
		\settowidth\labelwidth{\makelabel{#1}}%
		\setlength\leftmargin{\labelwidth}%
		\addtolength\leftmargin{\labelsep}}}%
	{\end{list}}

\newenvironment{customdescript}[1]
	{\begin{list}{}%
		{\renewcommand\makelabel[1]{\bfseries\color{mitred} ##1\hfill}%
		\settowidth\labelwidth{\makelabel{#1}}%
		\setlength\leftmargin{\labelwidth}%
		\addtolength\leftmargin{\labelsep}}}%
	{\end{list}}

\makeatletter

\newenvironment{customlist}[2]{
  \ifnum\@itemdepth >2\relax\@toodeep\else
      \advance\@itemdepth\@ne%
      \beamer@computepref\@itemdepth%
      \usebeamerfont{itemize/enumerate \beameritemnestingprefix body}%
      \usebeamercolor[fg]{itemize/enumerate \beameritemnestingprefix body}%
      \usebeamertemplate{itemize/enumerate \beameritemnestingprefix body begin}%
      \begin{list}
        {
            \usebeamertemplate{itemize \beameritemnestingprefix item}
        }
        { \leftmargin=#1 \itemindent=#2
            \def\makelabel##1{%
              {%  
                  \hss\llap{{%
                    \usebeamerfont*{itemize \beameritemnestingprefix item}%
                        \usebeamercolor[fg]{itemize \beameritemnestingprefix item}##1}}%
              }%  
            }%  
        }
  \fi
}
{
  \end{list}
  \usebeamertemplate{itemize/enumerate \beameritemnestingprefix body end}%
}
\makeatother

\newenvironment<>{varblock}[2][\textwidth]{%
  \setlength{\textwidth}{#1}
  \begin{actionenv}#3%
    \def\insertblocktitle{#2}%
    \par%
    \usebeamertemplate{block begin}}
  {\par%
    \usebeamertemplate{block end}%
  \end{actionenv}}

%% Notational commands:
\newcommand{\params}{\ensuremath{\xi}}
\newcommand{\vparms}{\ensuremath{\gvect{\params}}}
\renewcommand{\thefootnote}{\ensuremath{\fnsymbol{footnote}}}
\setcounter{footnote}{2}
\renewcommand{\thempfootnote}{\ensuremath{\fnsymbol{mpfootnote}}}
\newcommand{\newsubsection}[1]{\subsection{#1}\setcounter{subsection}{0}}


% Title Page
\title[]{Monte Carlo Project}
\author[]{22.901 Introduction to Computer Programming for Nuclear Engineers}
\institute[\insertpagenumber]{}
\date{January 30 - February 2, 2012} 

% -----------------------------------------------------------------------------
\begin{document}
% -----------------------------------------------------------------------------

% Inset title page
\frame{\titlepage}

%------------------------------------------------------------------------------
\begin{frame}{What is Monte Carlo?}

  \begin{itemize}
    \item Is a stochastic method to solve PDEs
    \vfill\item Useful for simulating random events
    \vfill\item Probability distributions capture physics
    \vfill\item Results are reported with a mean and variance
  \end{itemize}

\end{frame}
%------------------------------------------------------------------------------
\begin{frame}{Neutron Transport Equation}
  \scriptsize
  \begin{itemize}
    \item In its most detailed form (time-independent)
  \end{itemize}
  \vfill
\begin{multline*}
\overbrace{\underbrace{\hat{\Omega}\cdot\nabla\varphi\left(\vec{r},E,\hat{\Omega}\right)}_{leakage}+\underbrace{\Sigma_{t}\left(\vec{r},E\right)\varphi\left(\vec{r},E,\hat{\Omega}\right)}_{interactions}}^{destruction}=\\
\underbrace{\int_{0}^{4\pi}d^{2}\Omega^{\prime}\int_{0}^{\infty}dE^{\prime}\Sigma_{s}\left(\vec{r},E^{\prime}\rightarrow E,\hat{\Omega}^{\prime}\rightarrow\hat{\Omega}\right)\phi\left(\vec{r},E^{\prime}\right)}_{scattering\, production}+\\
\underbrace{\int_{0}^{4\pi}d^{2}\Omega^{\prime}\int_{0}^{\infty}dE^{\prime}\nu\Sigma_{f}\left(\vec{r},E^{\prime}\rightarrow E,\hat{\Omega}^{\prime}\rightarrow\hat{\Omega}\right)\phi\left(\vec{r},E^{\prime}\right)}_{fission\, production}+\underbrace{q\left(\vec{r},E,\hat{\Omega}\right)}_{external\, source}
\end{multline*}
  \vfill
  \begin{itemize}
    \item Simplified: 1-D homogeneous slab, 1 energy group, isotropic scattering in LAB, no fission, isotropic uniform source
  \end{itemize}
  \vfill
\[
\mu\frac{\partial\varphi}{\partial x}+\Sigma_{t}\varphi\left(x,\mu\right)=\Sigma_{s}\int_{-1}^{1}d\mu^{\prime}\varphi\left(x,\mu^{\prime}\rightarrow\mu\right)+\frac{Q}{2X} 
\]
\end{frame}
%------------------------------------------------------------------------------
\begin{frame}{Neutron Simulation}

  \begin{itemize}
    \item Monte Carlo neutronic analyses involve simulating neutrons 1-by-1
    \vfill\item So what happens in the life of a neutron
    \begin{enumerate}
      \vfill\item Neutron is born at some location and traveling angle
      \vfill\item It then transports so location - what can happen?
      \begin{itemize}
	\item Cross a surface - go back to 2
	\item Leak out of system - kill neutron
	\item Collide with an isotope
      \end{itemize}
      \vfill\item If it collides get a collision type
      \begin{itemize}
	\item Absorbed - kill neutron
	\item Scattered - get new angle in LAB
      \end{itemize}
      \vfill\item If particle is not dead go back to 2
    \end{enumerate}
  \end{itemize}
  \vfill
  \begin{center}\alert{That's it!}\end{center}
\end{frame}
%------------------------------------------------------------------------------
\begin{frame}{Source Code Layout}

  \begin{itemize}
    \item \texttt{main.f90} - The main program file
    \vfill \item \texttt{execute.f90} - A module where the neutrons life is simulated
    \vfill \item \texttt{global.f90} - A module that contains all of the global vars
    \vfill \item \texttt{geometry.f90} - A module that contains geometry information and procedures
    \vfill \item \texttt{material.f90} - A module that contains material information and procedures
    \vfill \item \texttt{particle.f90} - A module that contains particle information and procedures
    \vfill \item \texttt{tally.90} - A module that contains tally information and procedures
  \end{itemize}
  \vfill
  \alert{Make sure you copy OBJECTS, MAKEFILE, DEPENDENCIES and timing.f90 to your local source directory}

\end{frame}
%------------------------------------------------------------------------------
\begin{frame}{Module - \texttt{geometry.f90}}

  \begin{itemize}
    \item define geometry type with the following attributes
    \begin{itemize}
      \item Number of sub-slabs
      \vfill\item Length of slab
      \vfill\item Length of sub-slab
    \end{itemize}
    \vfill \item Contains one procedure - \texttt{read\_geometry}
    \begin{itemize}
      \item pass it a geometry type
      \vfill\item Read in two out of the 3 attributes
      \vfill\item calculate the 3rd attribute
    \end{itemize}
  \end{itemize}

\end{frame}
%------------------------------------------------------------------------------
\begin{frame}{Module - \texttt{material.f90}}

  \begin{itemize}
    \item define material type with the following attributes
    \begin{itemize}
      \item totalxs - total macroscopic cross section
      \vfill\item absxs - absorption macroscopic cross section
      \vfill\item scattxs - scattering macroscopic cross section
    \end{itemize}
    \vfill
    \[ \Sigma_{t} = \Sigma_{a} + \Sigma_{s} \]
    \vfill \item Contains one procedure - \texttt{read\_material}
    \begin{itemize}
      \item pass it a material type
      \vfill\item Read in each cross section from user
    \end{itemize}
  \end{itemize}

\end{frame}
%------------------------------------------------------------------------------
\begin{frame}{Module - \texttt{tally.f90}}

  \begin{itemize}
    \item define tally type with the following attributes (initialize to 0)
    \begin{itemize}
      \item \texttt{c1} - collision accumulator
      \item \texttt{c2} - square of collision accumulator
      \item \texttt{s1} - path accumulator
      \item \texttt{s2} - square of path accumulator
      \item \texttt{smean} - mean for tracklength est
      \item \texttt{cmean} - mean for collision est
      \item \texttt{svar} - variance for tracklength est
      \item \texttt{cvar} - variance for collision est
      \item \texttt{track} - temp. track var
      \item \texttt{coll} - temp. collision var
    \end{itemize}
    \vfill \item Contains the following procedures
    \begin{itemize}
      \item \texttt{sub: tally\_reset} with argument of a tally\_type
      \vfill\item \texttt{sub: bank\_tally} with argument of a tally\_type
      \vfill\item \texttt{sub: perform\_statistics} with arguments of a tally\_type, number of histories, sub-slab width
    \end{itemize}
  \end{itemize}

\end{frame}
%------------------------------------------------------------------------------
\begin{frame}{Module - \texttt{particle.f90}}

  \begin{itemize}
    \item use pdfs (in the static library)
    \vfill\item define particle type with the following attributes
    \begin{itemize}
      \item slab - the slab id number
      \vfill\item  xloc - the x location of the particle
      \vfill\item mu - the angle cosine of travel
      \vfill\item alive - a logical to indicate if the particle is alive
    \end{itemize}
    \vfill \item Contains one procedure - \texttt{particle\_init}
    \begin{itemize}
      \item pass it a particle\_type and length of slab
    \end{itemize}
  \end{itemize}

\end{frame}
%------------------------------------------------------------------------------
\begin{frame}{Module - \texttt{global.f90}}

  \begin{itemize}
    \item use all of the types listed on previous slides
    \vfill\item make sure you indicate that you want to save the variables
    \vfill\item define a geometry (geo), material (mat) and particle (neutron) type
    \vfill\item define an allocatable, 1-D, tally type called tal (it will be n\_slabs in length)
    \vfill\item define a variables for number of histories
    \vfill\item define a timer type see \texttt{timing.f90}
    \vfill\item this module contains two procedures
    \begin{itemize}
      \item sub: allocate\_problem - no args
      \item sub: free\_memory - no args
    \end{itemize}
  \end{itemize}
\end{frame}
%------------------------------------------------------------------------------
\begin{frame}{Module - \texttt{global.f90}}

  \begin{itemize}
    \item eventually will include a bunch of use commands
    \vfill\item no module-data here
    \vfill\item contains the following procedures
    \begin{itemize}
      \item \texttt{sub: run\_problem} - no args
      \item \texttt{sub: transport} - no args
      \item \texttt{sub: interaction} - no args
      \item \texttt{fun: get\_slab\_id} - no args
      \item \texttt{sub: reset\_tallies} - no args
      \item \texttt{sub: bank\_tallies} - no args
      \item \texttt{sub: print\_tallies - no args}
    \end{itemize}
  \end{itemize}

\end{frame}
%------------------------------------------------------------------------------
\begin{frame}{Program - \texttt{main.f90}}

  \begin{itemize}
    \item Put appropriate \textbf{use} commands - will be obvious from below
    \vfill\item Read in number of particles to simulate from user
    \vfill\item Read in geometry - call \texttt{read\_geometry}
    \vfill\item Read in material - call \texttt{read\_material}
    \vfill\item begin a timer
    \vfill\item allocate the problem - call \texttt{allocate\_problem}
    \vfill\item run the problem - call \texttt{run\_problem}
    \vfill\item stop the timer
    \vfill\item print results - call \texttt{printtallies}
    \vfill\item free memory - call \texttt{free\_memory}
    \vfill\item terminate the program
  \end{itemize}

\end{frame}
%------------------------------------------------------------------------------
\begin{frame}{Milestone 1}

  \begin{itemize}
    \item Develop the file structure for the code
    \vfill\item Follow description of modules on previous slides
    \vfill\item Metermine which procedures should be public and which should be private
    \vfill\item Make sure you do PARTIAL INCLUSION
    \vfill\item Goal is to read in the data and print it back out in print\_tallies routine
    \vfill\item Code will go into all subroutines but wont do anything
    \vfill\item Make sure timer works
  \end{itemize}
  \vfill
  \alert{If you have any trouble getting the code to work come see me at NW12-234 as we will be moving fast!}

\end{frame}
%------------------------------------------------------------------------------
\end{document}
